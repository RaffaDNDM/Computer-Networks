\chapter{Shell}

\section{Commands}

\begin{table}[h]
\begin{tabular}{|l|l|l|}
\hline
\multicolumn{2}{|l|}{\multirow{2}{*}{\textbf{man} man}}&{Shows info about man command and}\\
\multicolumn{2}{|l|}{} & {lists all the sections of the manual.}\\
\hline
\multicolumn{2}{|l|}{\textbf{strace} objFile} & {Lists all the system calls used in the program.}\\
\hline
\multicolumn{2}{|l|}{\textbf{gcc} -o objFile source \textbf{-v}} & {Lists all the path of libraries and headers used in creation of objFile.}\\
\hline
\multirow{3}{*}{\textbf{netstat}} & {-t} & {Lists all the active TCP connections showing domain names.}\\
\cline{2-3}
& {-u} & {Lists all the active UDP connections showing domain names.}\\
\cline{2-3}
& {-n} & {Lists all the active, showing IP and port numbers.}\\
\hline
\multicolumn{2}{|l|}{\textbf{nslookup} domain} & {Shows the IP address related to the domain (E.g. IP of www.google.it)}\\
\hline
\multicolumn{2}{|l|}{\multirow{2}{*}{\textbf{wc} [file]}} & {Prints in order newline, word, and byte counts for file}\\
\multicolumn{2}{|l|}{} & {if file not specified or equal to -, counts from stdin.}\\
\hline
\end{tabular}
\end{table}


\section{Files}

\begin{table}[h]
\begin{tabular}{|l|l|}
\hline
\multirow{2}{*}{\textbf{/etc/services}} & {List all the applications with their port}\\
& {and type of protocol (TCP/UDP).}\\
\hline
{\textbf{/usr/include/x86\_64-linux-gnu/bits/socket.h}} & {List all the protocol type possible for socket.}\\
\hline
{\textbf{/usr/include/x86\_64-linux-gnu/sys/socket.h}} & {Definition of struct sockaddr and specific ones.}\\
\hline
\end{tabular}
\end{table}

\section{vim}
\subsection{.vimrc}
In this section there will be shown the file \textbf{.vimrc} that can be put in the user home (\textbf{$\sim$} or \textbf{\$HOME} or \textbf{--}) or in the path \textbf{/usr/share/vim/} to change main settings of the program.

\lstinputlisting[caption={\footnotesize{web\_client.c}}, style=code, firstnumber=1, firstline=1, lastline=8, label=web_client, language=c]{Code/vimrc}

\subsection{Shortcuts}

%\section{Usefull information about shell}