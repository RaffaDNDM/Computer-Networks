\chapter{Shell}

\section{Commands}

\begin{table}[h]
\centering
\footnotesize
\begin{tabular}{|l|l|l|}
\hline
\multicolumn{2}{|l|}{\multirow{2}{*}{\textbf{man} man}}&{Shows info about man command and}\\
\multicolumn{2}{|l|}{} & {lists all the sections of the manual.}\\
\hline
\multicolumn{2}{|l|}{\textbf{strace} objFile} & {Lists all the system calls used in the program.}\\
\hline
\multicolumn{2}{|l|}{\textbf{ltrace} objFile} & {Lists all the library calls used in the program.}\\
\multicolumn{2}{|l|}{\textbf{gcc} -o objFile source \textbf{-v}} & {Lists all the path of libraries and headers used in creation of objFile.}\\
\hline
\multirow{3}{*}{\textbf{netstat}} & {-t} & {Lists all the active TCP connections showing domain names.}\\
\cline{2-3}
& {-u} & {Lists all the active UDP connections showing domain names.}\\
\cline{2-3}
& {-n} & {Lists all the active, showing IP and port numbers.}\\
\hline
\multicolumn{2}{|l|}{\textbf{nslookup} domain} & {Shows the IP address related to the domain (E.g. IP of www.google.it)}\\
\hline
\multicolumn{2}{|l|}{\multirow{5}{*}{\textbf{dig} @server name type}}&{DNS lookup utility.}\\
\multicolumn{2}{|l|}{}&{\textbf{server} name or IP address of the name server to query}\\
\multicolumn{2}{|l|}{}&{\textbf{name} name of the resource record that is to be looked up}\\
\multicolumn{2}{|l|}{}&{\textbf{type} type of query is required (ANY, A, MX, SIG, etc.)}\\
\multicolumn{2}{|l|}{}&{$\;\;\;\;\;\;\;\;\;\;$if no type is specified, A is performed by default}\\
\hline
\multicolumn{2}{|l|}{\multirow{2}{*}{\textbf{wc} [file]}} & {Prints in order newlines, words, and bytes (characters) counts for file}\\
\multicolumn{2}{|l|}{} & {if file not specified or equal to -, counts from stdin.}\\
\hline
\multicolumn{2}{|l|}{\multirow{2}{*}{\textbf{route} -n}} & {Show numerical addresses instead of trying to determine symbolic}\\
\multicolumn{2}{|l|}{} & {hostnames in routing table.}\\
\hline
\multicolumn{2}{|l|}{\multirow{2}{*}{\textbf{arp} -a}} & {List all the MAC addresses stored after some ARP}\\
\multicolumn{2}{|l|}{} & {requests and replies made by our ethernet interfaces.}\\
\hline
\end{tabular}
\end{table}


\section{UNIX Files}\label{files}
\begin{table}[h]
\centering
\footnotesize
\begin{tabular}{|l|l|}
\hline
\textbf{/etc/hosts} & {Local resolution table.}\\
\hline
\multirow{2}{*}{\textbf{/etc/services}} & {List all the applications with their port}\\
& {and type of protocol (TCP/UDP).}\\
\hline
{\textbf{/etc/protocols}} & {Internet protocols.}\\
\hline
{\textbf{/usr/include/x86\_64-linux-gnu/bits/socket.h}} & {List all the protocol type possible for socket.}\\
\hline
{\textbf{/usr/include/x86\_64-linux-gnu/sys/socket.h}} & {Definition of struct sockaddr and specific ones.}\\
\hline
\end{tabular}
\end{table}

%\section{Usefull information about shell}