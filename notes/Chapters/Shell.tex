\chapter{Shell}

\section{Commands}

\begin{table}[h]
\centering
\footnotesize
\begin{tabular}{|l|l|l|}
\hline
\multicolumn{2}{|l|}{\multirow{2}{*}{\textbf{man} man}}&{Shows info about man command and}\\
\multicolumn{2}{|l|}{} & {lists all the sections of the manual.}\\
\hline
\multicolumn{2}{|l|}{\textbf{strace} objFile} & {Lists all the system calls used in the program.}\\
\hline
\multicolumn{2}{|l|}{\textbf{gcc} -o objFile source \textbf{-v}} & {Lists all the path of libraries and headers used in creation of objFile.}\\
\hline
\multirow{3}{*}{\textbf{netstat}} & {-t} & {Lists all the active TCP connections showing domain names.}\\
\cline{2-3}
& {-u} & {Lists all the active UDP connections showing domain names.}\\
\cline{2-3}
& {-n} & {Lists all the active, showing IP and port numbers.}\\
\hline
\multicolumn{2}{|l|}{\textbf{nslookup} domain} & {Shows the IP address related to the domain (E.g. IP of www.google.it)}\\
\hline
\multicolumn{2}{|l|}{\multirow{2}{*}{\textbf{wc} [file]}} & {Prints in order newline, word, and byte counts for file}\\
\multicolumn{2}{|l|}{} & {if file not specified or equal to -, counts from stdin.}\\
\hline
\end{tabular}
\end{table}


\section{Files}

\begin{table}[h]
\centering
\footnotesize
\begin{tabular}{|l|l|}
\hline
\multirow{2}{*}{\textbf{/etc/services}} & {List all the applications with their port}\\
& {and type of protocol (TCP/UDP).}\\
\hline
{\textbf{/usr/include/x86\_64-linux-gnu/bits/socket.h}} & {List all the protocol type possible for socket.}\\
\hline
{\textbf{/usr/include/x86\_64-linux-gnu/sys/socket.h}} & {Definition of struct sockaddr and specific ones.}\\
\hline
\end{tabular}
\end{table}

\section{vim}
\subsection{.vimrc}
In this section there will be shown the file \textbf{.vimrc} that can be put in the user home (\textbf{$\sim$} or \textbf{\$HOME} or \textbf{--}) or in the path \textbf{/usr/share/vim/} to change main settings of the program.

\lstinputlisting[caption={\footnotesize{.vimrc}}, style=code, firstnumber=1, firstline=1, lastline=8, label=vimrc, language=c]{Code/vimrc}

\subsection{Shortcuts}

\begin{table}[h]
\centering
\footnotesize
\begin{tabular}{|l|l|}
\multicolumn{2}{c}{\textbf{Main}}\\
\hline
\multirow{2}{*}{\textbf{Esc}} & {Gets out of the current mode into the “command mode”.}\\
& {All keys are bound of commands}\\
\hline
\multirow{2}{*}{\textbf{i}} & {“Insert mode”}\\
& {for inserting text.}\\ 
\hline
\multirow{2}{*}{\textbf{:}} & {“Last-line mode”}\\ 
& {where Vim expects you to enter a command.}\\ 
\hline
\end{tabular}
\end{table}

\begin{table}[h]
\centering
\footnotesize
\begin{tabular}{|l|l|}
\multicolumn{2}{c}{\textbf{Navigation keys}}\\
\hline
\textbf{h}	& {moves the cursor one character to the left.}\\
\hline
{\textbf{j} or\textbf{ Ctrl + J}}	& {moves the cursor down one line.}\\
\hline
{\textbf{k} or \textbf{Ctrl + P}}	& {moves the cursor up one line.}\\
\hline
\textbf{l}  & {moves the cursor one character to the right.}\\
\hline
\textbf{0}	& {moves the cursor to the beginning of the line.}\\
\hline
\textbf{\$}	& {moves the cursor to the end of the line.}\\
\hline
\textbf{\^}	& {moves the cursor to the first non-empty character of the line}\\
\hline
\textbf{w}	& {move forward one word (next alphanumeric word)}\\
\hline
\textbf{W}	& {move forward one word (delimited by a white space)}\\
\hline
\textbf{5w}	& {move forward five words}\\
\hline
\textbf{b}	& {move backward one word (previous alphanumeric word)}\\
\hline
\textbf{B}	& {move backward one word (delimited by a white space)}\\
\hline
\textbf{5b}	& {move backward five words}\\
\hline
\textbf{G}	& {move to the end of the file}\\
\hline
\textbf{gg}	& {move to the beginning of the file.}\\
\hline
\end{tabular}
\end{table}

\begin{table}[h]
\centering
\footnotesize
\begin{tabular}{|l|l|}
\multicolumn{2}{c}{\textbf{Navigate around the document}}\\
\hline
\textbf{h}	& {moves the cursor one character to the left.}\\
\hline
\textbf{(}	& {jumps to the previous sentence}\\
\hline
\textbf{)}	& {jumps to the next sentence}\\
\hline
\textbf{$\lbrace$ }	& {jumps to the previous paragraph}\\
\hline
\textbf{$\rbrace$}	& {jumps to the next paragraph}\\
\hline
\textbf{[[}	& {jumps to the previous section}\\
\hline
\textbf{]]}	& {jumps to the next section}\\
\hline
\textbf{[]}	& {jump to the end of the previous section}\\
\hline
\textbf{][}	& {jump to the end of the next section}\\
\hline
\end{tabular}
\end{table}

\begin{table}[h]
\centering
\footnotesize
\begin{tabular}{|l|l|}
\multicolumn{2}{c}{\textbf{Insert text}}\\
\hline
\textbf{h}	& {moves the cursor one character to the left.}\\
\hline
\textbf{a}	& {Insert text after the cursor}\\
\hline
\textbf{A}	& {Insert text at the end of the line}\\
\hline
\textbf{i}	& {Insert text before the cursor}\\
\hline
\textbf{o}	& {Begin a new line below the cursor}\\
\hline
\textbf{O}	& {Begin a new line above the cursor}\\
\hline
\end{tabular}
\end{table}

\begin{table}[h]
\centering
\footnotesize
\begin{tabular}{|l|l|}
\multicolumn{2}{c}{\textbf{Special inserts}}\\
\hline
{\textbf{:r} [filename]} & {Insert the file [filename] below the cursor}\\
\hline
{\textbf{:r !}[command]} & {Execute [command] and insert its output below the cursor}\\
\hline
\end{tabular}
\end{table}

\begin{table}[h]
\centering
\footnotesize
\begin{tabular}{|l|l|}
\multicolumn{2}{c}{\textbf{Delete text}}\\
\hline
\textbf{x}	& {delete character at cursor}\\
\hline
\textbf{dw}	& {delete a word.}\\
\hline
\textbf{d0}	& {delete to the beginning of a line.}\\
\hline
\textbf{d\$} & {delete to the end of a line.}\\
\hline
\textbf{d)}	& {delete to the end of sentence.}\\
\hline
\textbf{dgg} & {delete to the beginning of the file.}\\
\hline
\textbf{dG}	& {delete to the end of the file.}\\
\hline
\textbf{dd}	& {delete line}\\
\hline
\textbf{3dd} & {delete three lines}\\
\hline
\end{tabular}
\end{table}

\begin{table}[h]
\centering
\footnotesize
\begin{tabular}{|l|l|}
\multicolumn{2}{c}{\textbf{Simple replace text}}\\
\hline
{\textbf{r}$\rbrace$text$\rbrace$}	& {Replace the character under the cursor with $\lbrace$text$\rbrace$}\\
\hline
\textbf{R}	& {Replace characters instead of inserting them}\\
\hline
\end{tabular}
\end{table}

\begin{table}[h]
\centering
\footnotesize
\begin{tabular}{|l|l|}
\multicolumn{2}{c}{\textbf{Copy/Paste text}}\\
\hline
\textbf{yy}	& {copy current line into storage buffer}\\
\hline
\textbf{["x]yy}& {	Copy the current lines into register x}\\
\hline
\textbf{p}	& {paste storage buffer after current line}\\
\hline
\textbf{P}	& {paste storage buffer before current line}\\
\hline
\textbf{["x]p}	& {paste from register x after current line}\\
\hline
\textbf{["x]P}	& {paste from register x before current line}\\
\hline
\end{tabular}
\end{table}


\begin{table}[h]
\centering
\footnotesize
\begin{tabular}{|l|l|}
\multicolumn{2}{c}{\textbf{Undo/Redo operation}}\\
\hline
\textbf{u}	& {undo the last operation.}\\
\hline
\textbf{Ctrl+r}	& {redo the last undo.}\\
\hline
\end{tabular}
\end{table}


\begin{table}[h]
\centering
\footnotesize
\begin{tabular}{|l|l|}
\multicolumn{2}{c}{\textbf{Search and Replace keys}}\\
\hline
\textbf{/search\_text}	& {search document for search\_text going forward}\\
\hline
\textbf{?search\_text}	& {search document for search\_text going backward}\\
\hline
\textbf{n} & {move to the next instance of the result from the search}\\
\hline
\textbf{N} & {move to the previous instance of the result}\\
\hline
\multirow{2}{*}{\textbf{:\%s/original/replacement}} & {Search for the first occurrence of the string “original”}\\
& {and replace it with “replacement”}\\
\hline
\multirow{2}{*}{\textbf{:\%s/original/replacement/g}} & {Search and replace all occurrences of the string}\\
& {“original” with “replacement”}\\
\hline
\multirow{2}{*}{\textbf{:\%s/original/replacement/gc}} & {Search for all occurrences of the string “original” but}\\
& {ask for confirmation before replacing them with “replacement”}\\
\hline
\end{tabular}
\end{table}

\begin{table}[h]
\centering
\footnotesize
\begin{tabular}{|l|l|}
\multicolumn{2}{c}{\textbf{Bookmarks}}\\
\hline
\textbf{m $\lbrace$a-z A-Z$\rbrace$} &	{Set bookmark $\lbrace$a-z A-Z$\rbrace$ at the current cursor position}\\
\hline
\textbf{:marks} & {List all bookmarks}\\
\hline
\textbf{'$\lbrace$a-z A-Z$\rbrace$}	 & {Jumps to the bookmark $\lbrace$a-z A-Z$\rbrace$}\\
\hline
\end{tabular}
\end{table}

\begin{table}[h]
\centering
\footnotesize
\begin{tabular}{|l|l|}
\multicolumn{2}{c}{\textbf{Select text}}\\
\hline
\textbf{v} & {Enter visual mode per character}\\
\hline
\textbf{V} & {Enter visual mode per line}\\
\hline
\textbf{Esc} & {Exit visual mode}\\
\hline
\end{tabular}
\end{table}

\begin{table}[h]
\centering
\footnotesize
\begin{tabular}{|l|l|}
\multicolumn{2}{c}{\textbf{Modify selected text}}\\
\hline
\textbf{~}	& {Switch case}\\
\hline
\textbf{d}	& {delete a word.}\\
\hline
\textbf{c}	& {change}\\
\hline
\textbf{y}	& {yank}\\
\hline
\textbf{>}	& {shift right}\\
\hline
\textbf{<}	& {shift left}\\
\hline
\textbf{!}	& {filter through an external command}\\
\hline
\end{tabular}
\end{table}

\begin{table}[h]
\centering
\footnotesize
\begin{tabular}{|l|l|}
\multicolumn{2}{c}{\textbf{Save and quit}}\\
\hline
\textbf{:q}	& {Quits Vim but fails when file has been changed}\\
\hline
\textbf{:w}	& {Save the file}\\
\hline
{\textbf{:w} new\_name} & {Save the file with the new\_name filename}\\
\hline
\textbf{:wq} & {Save the file and quit Vim.}\\
\hline
\textbf{:q!} & {Quit Vim without saving the changes to the file.}\\
\hline
\textbf{ZZ}	& {Write file, if modified, and quit Vim}\\
\hline
\textbf{ZQ}	& {Same as :q! Quits Vim without writing changes}\\
\hline
\end{tabular}
\end{table}

\clearpage
\subsection{Multiple files}
\begin{itemize}
\item{\textbf{Opening many files in the buffer}\\
\begin{center}
\begin{tabular}{c}
\begin{lstlisting}[linewidth=80pt, basicstyle=\footnotesize\sffamily,]
vim file1 file2
\end{lstlisting}
\end{tabular}
\end{center}
Launching this command, you can see only one file at the same time. To jump between the files you can use the following vim commands:
\begin{table}[h]
\centering
\footnotesize
\begin{tabular}{|l|l|}
\hline
\textbf{n(ext)} & {jumps to the next file}\\
\hline
\textbf{prev} & {jumps to the previous file}\\
\hline
\end{tabular}
\end{table}
}
\item{\textbf{Opening many files in several tabs}\\
\begin{center}
\begin{tabular}{c}
\begin{lstlisting}[linewidth=130pt, basicstyle=\footnotesize\sffamily,]
vim -p file1 file2 file3
\end{lstlisting}
\end{tabular}
\end{center}
All files will be opened in tabs instead of hidden buffers. The tab bar is displayed on the top of the editor.\\
You can also open a new tab with file \textit{filename} when you're already in Vim in the normal mode with command:
\begin{center}
\begin{tabular}{c}
\begin{lstlisting}[linewidth=80pt, basicstyle=\footnotesize\sffamily,]
:tabe filename
\end{lstlisting}
\end{tabular}
\end{center}
To manage tabs you can use the following vim commands:
\begin{table}[h]
\centering
\footnotesize
\begin{tabular}{|l|l|}
\hline
{\textbf{:tabn[ext]}   (command-line command)} & \multirow{2}{*}{Jumps to the next tab}\\
\cline{1-1}
{\textbf{gt}          (normal mode command)}&\\
\hline
\textbf{:tabp[revious] (command-line command)} & \multirow{2}{*}{Jumps to the previous tab}\\
\cline{1-1}
{\textbf{gT}          (normal mode command)}&\\
\hline
\multirow{2}{*}{\textbf{ngT}          (normal mode command)} & {Jumps to a specific tab index}\\
&{n= index of tab (starting by 1)}\\
\hline
{\textbf{:tabc[lose]} (command-line command)} & {Closes the current tab}\\
\hline
\end{tabular}
\end{table}
}
\item{\textbf{Open multiple files splitting the window}\\
\textit{splits the window horizontally}\\
\begin{center}
\begin{tabular}{c}
\begin{lstlisting}[linewidth=100pt, basicstyle=\footnotesize\sffamily,]
vim -o file1 file2
\end{lstlisting}
\end{tabular}
\end{center}
You can also split the window horizontally, opening the file \textit{filename}, when you're already in Vim in the normal mode with command:\\
\begin{center}
\begin{tabular}{c}
\begin{lstlisting}[linewidth=100pt, basicstyle=\footnotesize\sffamily,]
:sp[lit] filename
\end{lstlisting}
\end{tabular}
\end{center}
\textit{splits the window vertically}\\
\begin{center}
\begin{tabular}{c}
\begin{lstlisting}[linewidth=100pt, basicstyle=\footnotesize\sffamily,]
vim -O file1 file2
\end{lstlisting}
\end{tabular}
\end{center}
You can also split the window vertically, opening the file \textit{filename}, when you're already in Vim in the normal mode with command:\\
\begin{center}
\begin{tabular}{c}
\begin{lstlisting}[linewidth=100pt, basicstyle=\footnotesize\sffamily,]
:vs[plit] filename
\end{lstlisting}
\end{tabular}
\end{center}
Management of the windows can be done, staying in the normal mode of Vim, using the following commands:\\
\begin{table}[h]
\centering
\footnotesize
\begin{tabular}{|l|l|}
\hline
\textbf{Ctrl+w  <cursor-keys>} & \multirow{3}{*}{Jumps between windows}\\
\cline{1-1}
\textbf{Ctrl+w  [hjkl]} & {}\\
\cline{1-1}
\textbf{Ctrl+w  Ctrl+[hjkl]} & {}\\
\hline
\textbf{Ctrl+w  w} & \multirow{2}{*}{Jumps to the next window}\\
\cline{1-1}
\textbf{Ctrl+w  Ctrl+w} & {}\\
\hline
\textbf{Ctrl+w  W} & {Jumps to the previous window}\\
\hline
\textbf{Ctrl+w  p} & \multirow{2}{*}{Jumps to the last accessed window}\\
\cline{1-1}
\textbf{Ctrl+w  Ctrl+p} & {}\\
\hline
\textbf{Ctrl+w  c} & \multirow{2}{*}{Closes the current window}\\
\cline{1-1}
\textbf{:clo[se]} & {}\\
\hline
\textbf{Ctrl+w  o} & \multirow{2}{*}{Makes the current window the only one and closes all other ones}\\
\cline{1-1}
\textbf{:on[ly]} & {}\\
\hline
\end{tabular}
\end{table}
}
\end{itemize}

%\section{Usefull information about shell}