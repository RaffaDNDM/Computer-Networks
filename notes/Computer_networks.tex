\documentclass[a4paper, 8pt, twoside, openright]{book}
\usepackage[a4paper,top=4cm,bottom=2cm,left=2cm,right=2cm]{geometry}
\usepackage[english]{babel}
\usepackage[T1]{fontenc}
\usepackage[utf8]{inputenc}
\usepackage{fancyhdr}
\usepackage{float}
\usepackage{graphicx}
\usepackage{wrapfig}
\usepackage{siunitx} %per scrivere il simbolo °
\usepackage{verbatim} %per i commenti1
\usepackage{subfig}
\usepackage{amsmath}
\usepackage{algorithm}
\usepackage{algpseudocode}
\setcounter{secnumdepth}{3}
\setcounter{tocdepth}{6}
\usepackage{multirow}
\newcommand{\minitab}[2][l]{\begin{tabular}#1 #2\end{tabular}}
\usepackage{rotating}
\usepackage{xfrac}
\usepackage{cite}

\DeclareMathOperator*{\argmax}{arg\,max}
\DeclareMathOperator*{\argmin}{arg\,min}

%\usepackage{booktabs,array}
%\usepackage{tikz}

%\usepackage{tabularx}

%\usepackage{chngcntr}
%\counterwithin{table}{section}

%------------------------------ colors
\usepackage[usenames,dvipsnames,table]{xcolor} % use colors on table and more
\definecolor{333}{RGB}{51, 51, 51} % define custom color
\definecolor{background}{RGB}{248, 248, 255}
\definecolor{comment}{RGB}{17,167,5}
\definecolor{keyword}{RGB}{195,47,8}
\definecolor{string}{RGB}{142,195,0}
\definecolor{number}{RGB}{90,84,84}
\definecolor{identifier}{RGB}{0,90,201}

%------------------------------ source code
\usepackage{listings}

\lstset{
  basicstyle=\footnotesize\sffamily,
  commentstyle=\itshape\color{gray},
  captionpos=b,
  frame=shadowbox,
  language=HTML,
  rulesepcolor=\color{333},
  tabsize=2
}

\lstdefinestyle{code}{
  basicstyle=\footnotesize\sffamily,
  commentstyle=\color{comment},
}


%------------------------------ define Abstract environment, missing in the 'book' class
\newenvironment{abstract}{\cleardoublepage \null \vfill \begin{center}\bfseries\abstractname \end{center}}{\vfill\null}
\addto\captionsenglish{\renewcommand*\abstractname{Abstract}} % change Abstract title

%------------------------------ active url
\usepackage{url}
\renewcommand{\UrlFont}{\color{black}\small\ttfamily}
\usepackage{hyperref} % active ref
%------------------------------ macros
\newcommand{\sectionname}{Section} % define Section ref
\newcommand{\subsectionname}{Sub-section} % define Sub-section ref
\renewcommand*\arraystretch{1.4} % tables padding

%acronimi
\usepackage[printonlyused]{acronym}

\begin{document}
\frontmatter

\begin{titlepage} %------------------------------ TITLE PAGE
\begin{center}
\vbox to0pt{\vbox to\textheight{\vfill \includegraphics[width=11.5cm]{./Images/Background} \vfill}\vss}

\begin{center}
\begin{minipage}{.20\textwidth}
  \includegraphics[height=2.5cm]{./Images/Icon4}
\end{minipage}\begin{minipage}{.45\textwidth}
  \begin{table}[H]
  \begin{tabular}{l}
  \scshape{\Large{\bfseries{Padua University}}} \\
  \hline \\
  \scshape{\Large{Engineering Course}} \\
  \end{tabular}
  \end{table}
\end{minipage}
\end{center}


\vspace{1cm}
\emph{\Large{Master~of~Computer~Engineering}} \\
\vspace{0.9cm}
\scshape{\Large{\bfseries{Computer Networks}}} \\
\vspace{0.2cm} \linespread{1} \scshape{\large{\bfseries{}}}
\end{center}

\vspace{12cm}
\begin{center}
Raffaele Di Nardo Di Maio
\end{center}

\vfill
\begin{center}
\hspace{-0.2cm}
\line(1, 0){360}\\
\textsc{Accademic Year 2019-2020}
\end{center}
\end{titlepage}


\cleardoublepage % make left page blank
\thispagestyle{empty} %------------------------------ DEDICA

\begin{comment}
\null
\vspace{2cm}
\begin{flushright}
//DEDICA
\end{flushright}
\vfill
\begin{quote}
  \textit{}
\end{quote}
\vfill
\null
\end{comment}

%\begin{abstract} %------------------------------ ABSTRACT
%\addcontentsline{toc}{chapter}{Abstract}
%\markboth{}{} % remove header
%\thispagestyle{empty}
%This is the abstract
%\end{abstract}

%\input{Chapters/Abstract.tex}
\begingroup %------------------------------ CONTENTS
  \makeatletter
  \let\ps@plain\ps@empty
  \makeatother
  \tableofcontents  
  \clearpage
\endgroup

\mainmatter

\chapter{OSI model}
The \textit{Open System Interconnection (OSI)} is the basic standardization of concepts related to networks (Figure \ref{OSI}). It was made by Internet \textit{Standard Organization (ISO)}. Each computer, connected as a node in the network, needs to have all OSI functionalities.
\begin{figure}[h]
\centering
\includegraphics[scale=0.4]{Images/OSI/osi}\caption{\footnotesize{OSI model.}}\label{OSI}
\end{figure}

\section{Logical communication}
\begin{figure}[h]
\centering
\includegraphics[scale=0.3]{Images/OSI/logic}
\end{figure}
Layer 1 is the only one in which the real connection is also the logic connection. Each layer is a module (black-box) that implements functionality (see Section \ref{onion_section}).

\section{Control plane}
\begin{figure}[h]
\centering
\includegraphics[scale=0.4]{Images/OSI/request_AB}
\caption{\footnotesize{Request from A to B.}}\label{requestAB}
\end{figure}
\begin{figure}[h]
\centering
\includegraphics[scale=0.4]{Images/OSI/response_BA}
\caption{\footnotesize{Response from B to A.}}\label{responseBA}
\end{figure}
The control plane meaning comes from two words: "control" that is related to function activation and "plane", related to the geometry, because it's stacked in a sheet.\\
In OSI model, the \textit{direct connection} exists only between:
\begin{itemize}
\item{Upper and lower layers of the same device}
\item{Physical layers of different devices}
\end{itemize} 
From Figure \ref{requestAB} and Figure \ref{responseBA} we have seen two main types of function calls:
\begin{itemize}
\item{\textbf{Regular function calls}
\begin{itemize}
\item{library method invocations}
\item{system calls}
\item{HW enabled signals}
\end{itemize}
}
\item{\textbf{Callback functions}\\
the module of the upper layer is waken up by module of the lower layer.
\begin{itemize}
\item{OS signal handler\\
it asks library to call a function when something happens (EVENT-BASED PROGRAMMING)}
\item{Interrupt handlers}
\item{Blocking function calls\\
they start call but doesn't return if something doesn't happen}
\end{itemize}
}
\end{itemize}

\section{Data plane}
Data plane defines which data are shared among the network. Calling a function, we need to pass parameters to them (\textit{Data buffer}).\\
The PDU (Protocol Data Unit) of layer \textit{i+1} becomes the SDU (Service Data Unit), or payload, of lower Layer \textit{i}. Merging this payload, with the header of layer \textit{i}, we obtain the PDU of layer i (Figure \ref{pdu_sdu}). This procedure is called \textbf{encpsulation} (Figure \ref{encapsulation}).
\begin{figure}[h]
\centering
\includegraphics[scale=0.25]{Images/OSI/pdu_sdu}
\caption{PDU and SDU structure.}\label{pdu_sdu}
\end{figure}
\begin{figure}[h]
\centering
\includegraphics[scale=0.5]{Images/OSI/encapsulation}
\caption{\footnotesize{Encapsulation.}}\label{encapsulation}
\end{figure}
\vspace{3cm}
\section{Onion model}\label{onion_section}
The following image shows the layered structure of OS and computers and where OSI functionalities locations are highlighted. 
\begin{figure}[h]
\centering
\includegraphics[scale=0.5]{Images/OSI/onion}
\caption{\footnotesize{Onion model.}}\label{onion}
\end{figure}

\section{TCP/IP Architecture}
The TCP/IP architecture is a reorganization of the previously mentioned OSI model (Figure \ref{OSI}) and it composes the main structure of the Internet Protocol. 
\begin{figure}[h]
\centering
\includegraphics[scale=0.5]{Images/OSI/tcp_ip}
\end{figure}

\section{Application paradigms}
\subsection{Client-Server}
It's based on the presence of two main entities:
\begin{itemize}
\item{\textbf{Client =} active entity\\
it generates the request
}
\item{\textbf{Server =} passive entity\\
it's waiting for client requests and when it receives it, it only replies to it.
}
\end{itemize}
The main characteristic of this paradigm is the \textbf{"immediate" response time}, that is the time between the arrival of the request by the client and the reply with the generate response.\\
To send the request, the client needs to know:
\begin{itemize}
\item{server name}
\item{how to reach it}
\item{what data is required on server (trackable)}
\end{itemize}
\begin{figure}[h]
\centering
\includegraphics[scale=0.25]{Images/OSI/client_server}
\caption{\footnotesize{Client-Server architecture.}}\label{cs}
\end{figure}

\subsection{Peer-to-Peer (P2P)}
Its diffusion started at first years of XXI century. It's used to share media. Each node in the network can be client (making requests) or server (replying to requests).\\
In Figure \ref{p2p}, $USER_1$ doesn't know which is the user in the network that shared the content. Hence, he sends the request for the content to a node in the network and this one can reply with two possible responses:
\begin{itemize}
\item{\textbf{C=} content (media)}
\item{\textbf{R=} reference to another node (that has the required content or knows which node has the content)}
\end{itemize}
Each node can also forward the request to some other node and so it becomes the intermediary of the communication.
\begin{figure}[h]
\centering
\includegraphics[scale=0.35]{Images/OSI/p2p}
\caption{\footnotesize{P2P architecture.\\}}\label{p2p}
\end{figure}

\subsection{Publish/Subscribe/Notify}
The subscriber subscribes to the dispatcher (notifier) a set of messages that wants to be notified. The notifier usually filters the messages that it receives and, when there are new messages that respect the subscription of the user, notify them to the user.\\ The messages comes \textit{asynchronously} to the dispatcher. There is no \textit{Polling} made periodically by the user (there isn't Busy Waiting). There are some applications, like Whatsapp, that work in this way but in the past, this app made by Facebook doesn't really work asynchronously. In fact there was a polling policy.
\begin{figure}[H]
\centering
\includegraphics[scale=0.35]{Images/OSI/publish_subscribe}
\caption{\footnotesize{Publish/Subscribe/Notify architecture.}}\label{publish_subscribe}
\end{figure}

\section{Types of packets}
\begin{figure}[H]
\centering
\includegraphics[scale=0.28]{Images/OSI/packets}
\caption{\footnotesize{Standard names of packets.}}\label{packets}
\end{figure}
TCP connection works at Layer 4 but at upper layers, it seems to work as a stream. In TCP connection, it is usually specified the port number, that is the upper layer protocol specification (Layer 5).

\input{Chapters/Application_Layer}
\chapter{C programming}
The C is the most powerful language and also can be considered as the language nearest to Assembly language. Its power is the speed of execution and the easy interpretation of the memory.\\
C can be considered very important in Computer Networks because it doesn't hide the use of system calls. Other languages made the same thing, but hiding all the needs and evolution of Computer Network systems.

\section{Organization of data}\label{littleBig}
Data are stored in the memory in two possible ways, related to the order of bytes that compose it. There are two main ways, called also endians\ref{}.

\begin{comment}
\begin{figure}
\centering
\subfloat[][\emph{Mano con sfera riflettente}]{\includegraphics[width=.45\textwidth]{Sfera}} 

\quad\subfloat[][\emph{Belvedere}]{\includegraphics[width=.45\textwidth]{Belvedere}} \\
\caption{Esempio di figura composta da più sottofigure}\label{fig:subfig}\end{figure}
\end{comment}
 

\section{Types of data}


\section{Struct organization of memory}


\section{Structure of C program}

\chapter{Network services in C}\label{networkC}
\section{Application layer}
We need IP protocol to use Internet. In this protocol, level 5 and 6 are hidden in Application Layer.\\
In this case, Application Layer needs to interact with Transport Layer, that is implemented in OS Kernel (Figure \ref{app_kernel}). Hence Application and Transport can talk each other with System Calls.
\begin{figure}[h]
\centering
\includegraphics[scale=0.5]{Images/NetworkC/application}\caption{\footnotesize{System calls interface.}}\label{app_kernel}
\end{figure}

\section{socket()}\label{socket}
Entry-point (system call) that allow us to use the network services. It also allows application layer to access to level 4 of IP protocol. 
\begin{center}
\begin{tabular}{c}
\begin{lstlisting}[linewidth=270pt, basicstyle=\footnotesize\sffamily,]
#include <sys/types.h>
#include <sys/socket.h>

int socket(int domain, int type, int protocol);\\
\end{lstlisting}
\end{tabular}
\end{center}

\begin{table}[h]
\centering
\begin{tabular}{rcl}
\textbf{RETURN VALUE} & \multicolumn{2}{l}{\textit{File Descriptor (FD) of the socket} }\\
{} & \multicolumn{2}{l}{\textit{-1} if some error occurs and errno is set appropriately}\\
{} & \multicolumn{2}{l}{(You can check value of errno including <errno.h>).}\\
\end{tabular}
\end{table}

\begin{table}[h]
\centering
\begin{tabular}{rcl}
\textbf{domain =} & \multicolumn{2}{l}{\textit{Communication domain}}\\
{} & \multicolumn{2}{l}{protocol family which will be used for communication.}\\
{} & \textbf{AF\_INET:} & {IPv4 Internet Protocol}\\
{} & \textbf{AF\_INET6:} & {IPv6 Internet Protocol}\\
{} & \textbf{AF\_PACKET:} & {Low level packet interface}\\
& &\\
\textbf{type =} & \multicolumn{2}{l}{\textit{Communication semantics}}\\
{} & \textbf{SOCK\_STREAM:} & {Provides sequenced, reliable, two-way, connection-based}\\
{} & {} & {bytes stream. An OUT-OF-BAND data mechanism may}\\
{} & {} & {be supported.}\\
{} & \textbf{SOCK\_DGRAM} & {Supports datagrams (connectionless, unreliable messages} \\
& & {of a fixed maximum length).}\\
& & \\
\textbf{protocol =} & \multicolumn{2}{l}{\textit{Particular protocol to be used within the socket}}\\
{} & \multicolumn{2}{l}{Normally there is only a protocol for each socket type and protocol}\\
{} & \multicolumn{2}{l}{family (protocol=0), otherwise ID of the protocol you want to use}\\
\end{tabular}
\end{table}

\vspace{10cm}
\section{TCP connection}
In TCP connection, defined by type \textbf{SOCK\_STREAM} as written in the Section \ref{socket}, there is a client that connects to a server. It uses three primitives (related to File System primitives for management of files on disk) that do these logic actions:
\begin{enumerate}
\item{start (open bytes stream)}
\item{add/remove bytes from stream}
\item{finish (clos bytes stream)}
\end{enumerate}
TCP is used transfering big files on the network and for example with HTTP, that supports parallel download and upload (FULL-DUPLEX). The length of the stream is defined only at closure of the stream.
 
\subsection{Client}
\subsubsection{connect()}
The client calls \textbf{connect()} function, after \textbf{socket()} function of Section \ref{socket}. This function is a system call that client can use to define what is the remote terminal to which he wants to connect.

\begin{center}
\begin{tabular}{c}
\begin{lstlisting}[linewidth=370pt, basicstyle=\footnotesize\sffamily,]
#include <sys/types.h>
#include <sys/socket.h>

int connect(int sockfd, const struct sockaddr *addr,socklen_t addrlen);
\end{lstlisting}
\end{tabular}
\end{center}

\begin{table}[h]
\centering
\begin{tabular}{rcl}
\textbf{RETURN VALUE} & \multicolumn{2}{l}{\textit{0} if connection succeds}\\
{} & \multicolumn{2}{l}{\textit{-1} if some error occurs and errno is set appropriately}\\
& & \\
\textbf{sockfd =} & \multicolumn{2}{l}{\textit{Socket File Descriptor} returned by socket().}\\
& &\\
\textbf{addr =} & \multicolumn{2}{l}{\textit{Reference to struct sockaddr}}\\
{} & \multicolumn{2}{l}{sockaddr is a general structure that defines the concept of address.}\\
{} & \multicolumn{2}{l}{In practice it's a union of all the possible specific structures of each protocol.}\\
{} & \multicolumn{2}{l}{This approach is used to leave the function written in a generic way.}\\
& & \\
\textbf{addrlen =} & \multicolumn{2}{l}{\textit{Length of specific data structure used for sockaddr.}}\\
\end{tabular}
\end{table}
\vspace{8cm}
In the following there is the description of struct \textbf{sockaddr\_in}, that is the specific sockaddr structure implemented for family of protocls \textbf{AF\_INET}:

\begin{center}
\begin{tabular}{c}
\begin{lstlisting}[linewidth=350pt, basicstyle=\footnotesize\sffamily,]
#include <netinet/in.h>

struct sockaddr_in {
    sa_family_t    sin_family; /* address family: AF_INET */
    in_port_t      sin_port;   /* port in network byte order */
    struct in_addr sin_addr;   /* internet address */
};

/* Internet address. */
struct in_addr {
    uint32_t       s_addr;     /* address in network byte order */
};\\
\end{lstlisting}
\end{tabular}
\end{center}
The two addresses, needed to define a connection, are (see Figure \ref{addresses}):
\begin{itemize}
\item{\textbf{IP address (}$sin_addr$ in $sockaddr\_in\;\;struct$\textbf{)}\\
identifies a virtual interface in the network. It can be considered the entry-point for data arriving to the computer. \textit{It's unique in the world.}
}
\item{\textbf{Port number (}$sin_port$ in $sockaddr\_in\;\;struct$\textbf{)}\\
identifies to which application data are going to be sent. The port so must be open for that stream of data and it can be considered a service identifier. There are well known port numbers, related to standard services and others that are free to be used by the programmer for its applications (see Section \ref{files} to find which file contains well known port numbers). \textit{It's unique in the system.}
}
\end{itemize}
As mentioned in Section \ref{littleBig}, network data are organized as Big Endian, so in this case we need to insert the IP address according to this protocol. It can be done as in previous example or with the follow function:
\begin{center}
\begin{tabular}{c}
\begin{lstlisting}[linewidth=280pt, basicstyle=\footnotesize\sffamily,]
#include <sys/socket.h>
#include <netinet/in.h>
#include <arpa/inet.h>

int inet_aton(const char *cp, struct in_addr *inp);
\end{lstlisting}
\end{tabular}
\end{center}
The port number is written according to Big Endian architecture, through the next function:
\begin{center}
\begin{tabular}{c}
\begin{lstlisting}[linewidth=200pt, basicstyle=\footnotesize\sffamily,]
#include <arpa/inet.h>

uint16_t htons(uint16_t hostshort);
\end{lstlisting}
\end{tabular}
\end{center}
\begin{figure}[h]
\centering
\includegraphics[scale=0.4]{Images/NetworkC/addresses}\caption{\footnotesize{After successful connection.}}\label{addresses}
\end{figure}

\subsubsection{write()}
Application protocol uses a readable string, to excange readable information (as in HTTP). This tecnique is called simple protocol and commands, sent by the protocol, are standardized and readable strings.  

\begin{center}
\begin{tabular}{c}
\begin{lstlisting}[linewidth=280pt, basicstyle=\footnotesize\sffamily,]
#include <unistd.h>

ssize_t write(int fd, const void *buf, size_t count);
\end{lstlisting}
\end{tabular}
\end{center}

\begin{table}[h]
\centering
\begin{tabular}{rcl}
\textbf{RETURN VALUE} & \multicolumn{2}{l}{\textit{Number of bytes written} on success}\\
{} & \multicolumn{2}{l}{\textit{-1} if some error occurs and errno is set appropriately}\\
& & \\
\textbf{fd =} & \multicolumn{2}{l}{\textit{Socket File Descriptor} returned by socket().}\\
& &\\
\textbf{buf =} & \multicolumn{2}{l}{\textit{Buffer of characters to write}}\\
& & \\
\textbf{count =} & \multicolumn{2}{l}{\textit{Max number of bytes to write} in the file (stream).}\\
\end{tabular}
\end{table}
The write buffer is usually a string but we don't consider the null value (\textbf{$\backslash 0$} character), that determine the end of the string, in the evaluation of count (\textbf{strlen(buf)-1}). This convention is used because \textbf{$\backslash 0$} can be part of characters stream.\\

\subsubsection{read()}
The client uses this blocking function to wait and obtain response from the remote server. Not all the request are completed immediat from the server, for the meaning of stream type of protocol. Infact in this protocol, there is a flow for which the complete sequence is defined only at the closure of it\ref{socket}.\\
\textbf{read()} is consuming bytes fom the stream asking to level 4 a portion of them, because it cannot access directly to bytes in Kernel buffer. Lower layer controls the stream of information that comes from the same layer of remove system.\\
\vspace{3cm}
\begin{center}
\begin{tabular}{c}
\begin{lstlisting}[linewidth=280pt, basicstyle=\footnotesize\sffamily,]
#include <unistd.h>

       ssize_t read(int fd, void *buf, size_t count);
\end{lstlisting}
\end{tabular}
\end{center}

\begin{table}[h]
\centering
\begin{tabular}{rcl}
\textbf{RETURN VALUE} & \multicolumn{2}{l}{\textit{Number of bytes read} on success}\\
{} & \multicolumn{2}{l}{\textit{0} if EOF is reached (end of the stream)}\\
{} & \multicolumn{2}{l}{\textit{-1} if some error occurs and errno is set appropriately}\\
& & \\
\textbf{fd =} & \multicolumn{2}{l}{\textit{Socket File Descriptor} returned by socket().}\\
& &\\
\textbf{buf =} & \multicolumn{2}{l}{\textit{Buffer of characters in which it reads and stores info}}\\
& & \\
\textbf{count =} & \multicolumn{2}{l}{\textit{Max number of bytes to read} from the file (stream).}
\end{tabular}
\end{table}
So if \textbf{read()} doesn't return, this means that the stream isn't ended but the system buffer is empty.\\
If \textbf{read=0}, the function met EOF and the local system buffer is now empty. This helps client to understand that server ended before the connection.

\begin{figure}[h]
\centering
\includegraphics[scale=0.5]{Images/NetworkC/read_write1}\caption{\footnotesize{Request by the client.}}\label{rw1}
\end{figure}

\begin{figure}[h]
\centering
\includegraphics[scale=0.5]{Images/NetworkC/read_write2}\caption{\footnotesize{Response from the server.}}\label{rw2}
\end{figure}

\clearpage
\subsection{Server}
A server is a daemon, an application that works in background forever. The end of this process can be made only through the use of the Operating System.
\subsubsection{bind()}
\begin{center}
\begin{tabular}{c}
\begin{lstlisting}[linewidth=350pt, basicstyle=\footnotesize\sffamily,]
#include <sys/types.h>
#include <sys/socket.h>

int bind(int sockfd, const struct sockaddr *addr, socklen_t addrlen);
\end{lstlisting}
\end{tabular}
\end{center}

\begin{table}[h]
\centering
\begin{tabular}{rcl}
\textbf{RETURN VALUE} & \multicolumn{2}{l}{\textit{0} on success}\\
{} & \multicolumn{2}{l}{\textit{-1} if some error occurs and errno is set appropriately}\\
{} & \multicolumn{2}{l}{(You can check value of errno including <errno.h>).}\\
& & \\
\textbf{sockfd =} & \multicolumn{2}{l}{\textit{Socket File Descriptor} returned by socket().}\\
& &\\
\textbf{addr =} & \multicolumn{2}{l}{\textit{Reference to struct sockaddr}}\\
{} & \multicolumn{2}{l}{sockaddr is a general structure that defines the concept of address.}\\
& & \\
\textbf{addrlen =} & \multicolumn{2}{l}{\textit{Length of specific data structure used for sockaddr.}}\\
\end{tabular}
\end{table}

\subsubsection{listen()}
\begin{center}
\begin{tabular}{c}
\begin{lstlisting}[linewidth=185pt, basicstyle=\footnotesize\sffamily,]
#include <sys/types.h>
#include <sys/socket.h>

int listen(int sockfd, int backlog);
\end{lstlisting}
\end{tabular}
\end{center}

\begin{table}[h]
\centering
\begin{tabular}{rcl}
\textbf{RETURN VALUE} & \multicolumn{2}{l}{\textit{0} on success}\\
{} & \multicolumn{2}{l}{\textit{-1} if some error occurs and errno is set appropriately}\\
{} & \multicolumn{2}{l}{(You can check value of errno including <errno.h>).}\\
& & \\
\textbf{sockfd =} & \multicolumn{2}{l}{\textit{Socket File Descriptor} returned by socket().}\\
& &\\
\textbf{backlog =} & \multicolumn{2}{l}{\textit{Maximum length of queue of pending connections}}\\
& {The number of pending connections for sockfd can grow up}\\
& {to this value.}\\
& {The normal distribution of new requests by clients}\\
& {is usually Poisson, organized as in Figure \ref{poisson}.}
\end{tabular}
\end{table}
The listening socket, identified by \textbf{sockfd}, is unique for each association of a port number and a IP address of the server (Figure \ref{listen}).

\begin{figure}[h]
\centering
\includegraphics[scale=0.3]{Images/NetworkC/poisson}\caption{\footnotesize{Poisson distribution of connections by clients.}}\label{poisson}
\end{figure}

\begin{figure}[h]
\centering
\includegraphics[scale=0.3]{Images/NetworkC/listen}\caption{\footnotesize{listen() function.}}\label{listen}
\end{figure}

\subsubsection{accept()}
\begin{center}
\begin{tabular}{c}
\begin{lstlisting}[linewidth=340pt, basicstyle=\footnotesize\sffamily,]
#include <sys/types.h>
#include <sys/socket.h>

int accept(int sockfd, struct sockaddr *addr, socklen_t *addrlen);
\end{lstlisting}
\end{tabular}
\end{center}

\begin{table}[h]
\centering\footnotesize
\begin{tabular}{rl}
\textbf{RETURN VALUE} & {\textit{Accepted Socket Descriptor}}\\
{} & {it will be used by server, to manage requests and responses from}\\
{} & {that specific client.}\\
{} & {\textit{-1} if some error occurs and errno is set appropriately}\\
{} & {(You can check value of errno including <errno.h>).}\\
& \\
\textbf{sockfd =} & {\textit{Listen Socket File Descriptor}}\\
&\\
\textbf{addr =} & {\textit{Reference to struct sockaddr}}\\
{} & {It's going to be filled by the accept() function.}\\
& \\
\textbf{addrlen =} & {\textit{Length of the struct of addr.}}\\
{} & {It's going to be filled by accept() function.}\\
{} & {( accept() is used in different cases so it can return different}\\
{} & {type of specific implementation of struct addr.)}\\
&\\
\end{tabular}
\end{table}

To manage many clients requests, we use the \textbf{accept()} function to extablish the connection one-to-one with each client, creating a uniquely socket with each client.\\
This function extracts the  first   connection request on the queue of pending connections for the listening socket \textbf{sockfd} creates a new connected socket, and returns a new file descriptor  referring  to that socket. The accept() is blocking for the server when the queue of pending requests is empty (Figure \ref{pending_accept}).\\
At lower layers of ISO/OSI, the port number and the IP Address are the same identifiers, to which listening socket is associated (Figure \ref{accept}).
 
\begin{figure}[h]
\centering
\includegraphics[scale=0.3]{Images/NetworkC/accept}\caption{\footnotesize{accept() function.}}\label{accept}
\end{figure}

\begin{figure}[h]
\centering
\includegraphics[scale=0.4]{Images/NetworkC/pending_accept}\caption{\footnotesize{Management of pending requests with accept().}}\label{pending_accept}
\end{figure}

\section{UDP connection}
UDP connection is defined by type \textbf{SOCK\_DGRAM} as specified in Section \ref{socket}. It's used for application in which we use small packets and we want immediate feedback directly from application. It isn't reliable because it doesn't need confirmation in transport layer. It's used in Twitter application and in video streaming.
\chapter{Gateway}
A gateway is a device that forwards messages from another device, the client, to a second device, the server or another gateway.
In the following figures, there are two examples of gateways: Layer-3 gateways (routers in Section \ref{router_section}) and Layer-7 gateways (proxy).
\begin{figure}[h]
\centering
\includegraphics[scale=0.4]{Images/Gateway/gateway_3}
\caption{\footnotesize{Router (Layer-3 gateway).}}\label{gateway_3}
\end{figure}
\begin{figure}[h]
\centering
\includegraphics[scale=0.4]{Images/Gateway/gateway_7}
\caption{\footnotesize{Proxy (Layer-7 gateway).}}\label{gateway_7}
\end{figure}

\section{Proxy}
A Layer-7 gateway is also called proxy. It works as an intermediary between two identical protocols (Figure \ref{proxy}). Instead of Layer-3 gateways, proxy can also see the full stream of data, analyze HTTP headers and implement new functions. 
\begin{figure}[h]
\centering
\includegraphics[scale=0.42]{Images/Gateway/proxy}
\caption{\footnotesize{Example of proxy use.}}\label{proxy}
\end{figure}
The main possible functions are:
\begin{itemize}
\item{\textbf{Caching}\\
It's used to reduce traffic directed to the server. The proxy does the most expensive job, managing all the requests of the same page of the server. \\
After the request of the page for the first time, the proxy asks the page to the server and then stores in its system, before replying. Hence the next clients requests of the same page will be manage only by proxy because the page was already stored in its system.\\
In this case the server needs to manage only a request by proxy and provide a response to proxy.
\begin{figure}[h]
\centering
\includegraphics[scale=0.38]{Images/Gateway/proxy_cache_no}
\caption{\footnotesize{Example of caching without proxy.}}\label{proxy_cache_no}
\end{figure}
\begin{figure}[h]
\centering
\includegraphics[scale=0.4]{Images/Gateway/proxy_cache}
\caption{\footnotesize{Example of caching using proxy.}}\label{proxy_cache}
\end{figure}
}
\item{\textbf{Filtering}\\
The proxy can do two actions:
\begin{itemize}
\item{\textbf{Filtering the requested resource by the client}\\
there are many companies that doesn't give access to some services (E.g. no access to Facebook, Youtube, ...).\\
We cannot use a filtering approach at lower levels because in some cases clients can access to services through intermediate addresses, different from the one we want to reach. Hence we need to analyze the HTTP request at upper layer.}
\item{\textbf{Filtering the content of the response}\\
for parent control approach.}
\end{itemize}
\begin{figure}[h]
\centering
\includegraphics[scale=0.4]{Images/Gateway/proxy_filter}
\caption{\footnotesize{Example of proxy filtering.}}\label{proxy_filter}
\end{figure}
}
\item{\textbf{Web Application Firewall (WAF)}\\
The proxy is specialized and used to block suspicious requests. This is done by analyzing request content, looking for not secure pattern.\\
A possible pattern can be \textit{".."} in the path of the resource, that could give access to not accessible part of the File System (injection). Another possible pattern could be a suspicious parameter for a web application to manage SQL database (SQL injection).
\begin{figure}[h]
\centering
\includegraphics[scale=0.4]{Images/Gateway/proxy_waf}
\caption{\footnotesize{Example of WAF use.}}\label{proxy_waf}
\end{figure}
}
\item{\textbf{Load Balancing}\\
The proxy is a load balancer for the clients requests to the server.\\
There are many servers to manage requests by client. The client makes the request of the web page but in the reality it's talking with the proxy, that manage the request by sending it to a particular server.\\
This action is repeated for each client's request. Hence the client thinks that is talking to one server but in reality, the proxy distribute the requests among several servers. 
\begin{figure}[h]
\centering
\includegraphics[scale=0.4]{Images/Gateway/proxy_load}
\caption{\footnotesize{Example of load balancing through proxy.}}\label{proxy_load}
\end{figure}
}
\end{itemize}

\section{Router} \label{router_section}
A router is a device that does two main functions:
\begin{enumerate}
\item{\textbf{Routing}\\
it decides on which outbound link send the packet. This decision is based on destination address and its router table (Table \ref{routing_table}). In each routing table, a network address is associated to an outbound interface, where the packet will be forwarded.\\
Each network address is followed by a \textbf{"/"} and a number that defines how many most significant bits of \textbf{net mask} are set to 1. The default adddress, that is always in each routing table, is \textbf{0.0.0.0}. This one is associated to the interface on which the packet will be sent if no one of the previous messages matches with the one of the destination.\\
For each entry of the routing table, the network address is ANDed with its net mask and the IP address, we are looking for, ANDed with that net mask gives us the same result of the first one, the packet is sent to the corresponding interface.\\
The default address \textbf{0.0.0.0} is associated with a net mask, composed by all 0's. Hence every address, ANDed with this net mask, matches with default address \textbf{0.0.0.0}.
\begin{table}[h]
\centering \footnotesize
\begin{tabular}{|c|c|}
\hline
\textbf{Address prefix} & \textbf{Outbound interface}\\
\hline
{147.162.0.0\textit{/16}} & {2}\\
{88.80.187.0\textit{/24}} & {4}\\
{...} & {...}\\
{0.0.0.0} & {1}\\
\hline
\end{tabular}
\caption{Example of a routing table.}\label{routing_table}
\end{table}
}
\item{\textbf{Switching}\\
it sends the packet to the link previously selected.}
\end{enumerate}
Each router manages all the incoming packets, storing them in a input \textbf{FIFO buffer} (\textit{Standard Service Local}).By default, if packets arrive too fast to in the buffer, w.r.t. velocity of incoming data processing, new packets are dropped if buffer is already full according to some policy (Figure \ref{}).\\
Hence routers has not responsability if some packets are dropped because of it declares it in advance and its goal is to give user the best effort. The behaviour of the router management of the input buffer is based on different policy, according to a goal:\\
\begin{itemize}
\item{\textit{To reduce} \textbf{latency}\\
the packets are sorted by precedence index
}
\item{\textit{To reduce} \textbf{loss rate}\\
dropped packets are the last enetered without \textit{R} bit set
}
\item{\textit{To reduce} \textbf{throughput}\\
the packets are stored by index, calculated by the router, based on the amount of data transfered from each source/destination in a time unit (e.g. RSUP, virtual clock, MPLS, Stop \& GO criteria)
}
\end{itemize}
The user cannot set all the possible criteria, because these depend from agreement developed with Service Provider. Hence the Internet Service Provider, if all criteria are set, reset them all before sending packets to Internet.
\chapter{Layer 3: }
The Internet protocol was the result of research job made by american Department of Defence (DoD). \textit{Internet} means Inter-networks communication and was designed for use of interconnected systems of packet-switched computer communication networks. The only things in common between the networks is the packet orientation.\\
The internet protocol provides for transmitting blocks of data called datagrams from sources to destinations, where sources and destinations are hosts identified by fixed length addresses.  The internet protocol also provides for fragmentation and reassembly of long datagrams, if necessary, for transmission through "small packet" networks.\\

Ping is the most known service of Layer 3 \cite{RFC791}.
\chapter{HTTP protocol}
HTTP protocol was presented for the first time in the RFC 1945 (Request for Comment).\\
The Hypertext Transfer Protocol (HTTP) is an application-level protocol with the lightness and speed necessary for distributed, collaborative, hypermedia information systems. It is a generic, stateless, object-oriented protocol which can be used for many tasks, such as name servers and distributed object management systems, through extension of its request methods (commands).\\
It's not the first Hypertext protocol in history because there was Hypertalk, made by Apple before. \\
A feature of HTTP is the typing of data representation, allowing systems to be built independently f the data being transferred. HTTP has been in use by the World-Wide Web global information initiative since 1990.

\section{Terminology}
\begin{itemize}
\item{\textbf{connection}\\
a transport layer virtual circuit established between two application programs for the purpose of communication.}
\item{\textbf{message}\\
the basic unit of HTTP communication, consisting of a structured sequence of octets matching the syntax defined in Section 4 and transmitted via the connection.}
\item{\textbf{request}\\
an HTTP request message.
}
\item{\textbf{response}\\
an HTTP response message.}
\item{\textbf{resource}\\
a network data object or service which can be identified by a URI.}
\item{\textbf{entity}\\
a particular representation or rendition of a data resource, or reply from a service resource, that may be enclosed within a request or response message. An entity consists of metainformation in the form of entity headers and content in the form of an entity body.}
\item{\textbf{client}\\
an application program that establishes connections for the purpose of sending requests.}
\item{\textbf{user agent}\\
the client which initiates a request. These are often browsers, editors, spiders (web-traversing robots), or other end user tools.}
\item{\textbf{server}\\
an application program that accepts connections in order to service requests by sending back responses.}
\item{\textbf{origin server}\\
the server on which a given resource resides or is to be created.}
\item{\textbf{proxy}\\
an intermediary program which acts as both a server and a client for the purpose of making requests on behalf of other clients. Requests are serviced internally or by passing them, with possible translation, on to other servers. A proxy must interpret and, if necessary, rewrite a request message before forwarding it.\\
Proxies are often used as client-side portals
through network firewalls and as helper applications for handling requests via protocols not implemented by the user agent.}
\item{\textbf{gateway}\\
a server which acts as an intermediary for some other server. Unlike a proxy, a gateway receives requests as if it were the origin server for the requested resource; the requesting client may not be aware that it is communicating with a gateway.\\
Gateways are often used as server-side portals through network firewalls and as protocol translators for access to resources stored on non-HTTP systems.}
\item{\textbf{tunnel}\\
a tunnel is an intermediary program which is acting as a blind relay between two connections. Once active, a tunnel is not considered a party to the HTTP communication, though the tunnel may have been initiated by an HTTP request. The tunnel ceases to exist when both ends of the relayed connections are closed.\\
Tunnels are used when a portal is necessary and the intermediary cannot, or should not, interpret the relayed communication.}
\item{\textbf{cache}\\
a program's local store of response messages and the subsystem that controls its message storage, retrieval, and deletion. A cache stores cachable responses in order to reduce the response time and network bandwidth consumption on future, equivalent requests. Any client or server may include a cache, though a cache cannot be used by a server while it is acting as a tunnel.}
\end{itemize}
Any given program may be capable of being both a client and a server; our use of these terms refers only to the role being performed by the program for a particular connection, rather than to the program's capabilities in general. Likewise, any server may act as an origin server, proxy, gateway, or tunnel, switching behavior based on the nature of each request.

\section{Basic rules}
The following rules are used throughout are used to describe the grammar used in the RFC 1945.
\begin{table}[h]
\centering
\footnotesize
\begin{tabular}{rl}
\textbf{OCTET =}& <any 8-bit sequence of data>\\
\textbf{CHAR =}& <any US-ASCII character (octets 0 - 127)>\\
\textbf{UPALPHA =}& <any US-ASCII uppercase letter "A".."Z">\\
\textbf{LOALPHA =}& <any US-ASCII lowercase letter "a".."z">\\
\textbf{ALPHA =}& UPALPHA | LOALPHA\\
\textbf{DIGIT =}& <any US-ASCII digit "0".."9">\\
\textbf{CTL =}& <any US-ASCII control character (octets 0 - 31) and DEL (127)>\\
\textbf{CR =}& <US-ASCII CR, carriage return (13)>\\
\textbf{LF =}& <US-ASCII LF, linefeed (10)>\\
\textbf{SP =}& <US-ASCII SP, space (32)>\\
\textbf{HT =}& <US-ASCII HT, horizontal-tab (9)>\\
\textbf{<"> =}& <US-ASCII double-quote mark (34)>\\
\end{tabular}
\end{table}

\section{Messages}
\subsection{Different versions of HTTP protocol}
\begin{itemize}
\item{\textbf{HTTP/0.9 Messages}\\
Simple-Request and Simple-Response do not allow the use of any header information and are limited to a single request method (GET).\\ Use of the Simple-Request format is discouraged because it prevents the server from identifying the media type of the returned entity.
\begin{center}
\begin{tabular}{c}
\begin{lstlisting}[linewidth=240pt, basicstyle=\footnotesize\sffamily,]
HTTP-message = Simple-Request | Simple-Response
\end{lstlisting}
\end{tabular}
\end{center}
\begin{center}
\begin{tabular}{c}
\begin{lstlisting}[linewidth=230pt, basicstyle=\footnotesize\sffamily,]
Simple-Request  = "GET" SP Request-URI CRLF


Simple-Response = [ Entity-Body ]
\end{lstlisting}
\end{tabular}
\end{center}
}
\item{\textbf{HTTP/1.0 Messages}\\
Full-Request and Full-Response use the generic message format of RFC 822 for transferring entities. Both messages may include optional header fields (also known as "headers") and an entity body. The entity body is separated from the headers by a null line (i.e., a line with nothing preceding the CRLF).
\begin{center}
\begin{tabular}{c}
\begin{lstlisting}[linewidth=230pt, basicstyle=\footnotesize\sffamily,]
HTTP-message = Full-Request | Full-Response
\end{lstlisting}
\end{tabular}
\end{center}
\begin{center}
\begin{tabular}{c}
\begin{lstlisting}[linewidth=340pt, basicstyle=\footnotesize\sffamily,]
Full-Request = Request-Line
               *(General-Header | Request-Header | Entity-Header)
               CRLF
               [Entity-Body]
               
               
Full-Response = Status-Line
                *(General-Header | Request-Header | Entity-Header)
                CRLF
                [Entity-Body]               
\end{lstlisting}
\end{tabular}
\end{center}
}
\end{itemize}

\subsection{Headers}
The order in which header fields are received is not significant. However, it is "good practice" to send General-Header fields first, followed by Request-Header or Response-Header fields prior to the Entity-Header fields.\\
Multiple HTTP-header fields with the same field-name may be present in a message if and only if the entire field-value for that header field is defined as a comma-separated list.
\begin{center}
\begin{tabular}{c}
\begin{lstlisting}[linewidth=260pt, basicstyle=\footnotesize\sffamily,]
HTTP-header = field-name ":" [ field-value ] CRLF
\end{lstlisting}
\end{tabular}
\end{center}

\subsection{Request-Line}
\begin{center}
\begin{tabular}{c}
\begin{lstlisting}[linewidth=320pt, basicstyle=\footnotesize\sffamily,]
Request-Line = Method SP Request-URI SP HTTP-Version CRLF

Method         = "GET" | "HEAD" | "POST" | extension-method

extension-method = token
\end{lstlisting}
\end{tabular}
\end{center}
The list of methods acceptable by a specific resource can change dynamically; the client is notified through the return code of the response if a method is not allowed on a resource.\\
Servers should return the status code 501 (not implemented) if the method is unrecognized or not implemented.

\subsection{Request-URI}
The Request-URI is a Uniform Resource Identifier and identifies the resource upon which to apply the request.
\begin{center}
\begin{tabular}{c}
\begin{lstlisting}[linewidth=190pt, basicstyle=\footnotesize\sffamily,]
Request-URI = absoluteURI | abs_path
\end{lstlisting}
\end{tabular}
\end{center}
The absoluteURI form is only allowed when the request is being made to a proxy. The proxy is requested to forward the request and return the response. If the request is GET or HEAD and a prior response is cached, the proxy may use the cached message if it passes any restrictions in the Expires header field.\\
Note that the proxy may forward the request on to another proxy or directly to the server specified by the absoluteURI. In order to avoid request loops, a proxy must be able to recognize all of its server names, including any aliases, local variations, and the numeric IP address.\\\\
The most common form of Request-URI is that used to identify a resource on an origin server or gateway. In this case, only the absolute path of the URI is transmitted.

\subsection{Request Header}
The request header fields allow the client to pass additional information about the request, and about the client itself, to the server.\\
These fields act as request modifiers, with semantics equivalent to the parameters on a programming language method (procedure) invocation.
\begin{center}
\begin{tabular}{c}
\begin{lstlisting}[linewidth=410pt, basicstyle=\footnotesize\sffamily,]
Request-Header = Authorization | From | If-Modified-Since | Referer | User-Agent
\end{lstlisting}
\end{tabular}
\end{center} 

\subsection{Status line}
\begin{center}
\begin{tabular}{c}
\begin{lstlisting}[linewidth=330pt, basicstyle=\footnotesize\sffamily,]
Status-Line = HTTP-Version SP Status-Code SP Reason-Phrase CRLF
\end{lstlisting}
\end{tabular}
\end{center}
\begin{table}[h]
\centering
\footnotesize
\begin{tabular}{|r|l|}
\multicolumn{2}{c}{\textbf{General Status code}}\\
\hline
\textbf{1xx: Informational} & {Not used, but reserved for future use}\\
\hline
\textbf{2xx: Success}&{The action was successfully received,}\\
\hline
& {understood, and accepted.}\\
\hline
\textbf{3xx: Redirection} & {Further action must be taken in order to}\\
&{complete the request}\\
\hline
\textbf{4xx: Client Error}&{The request contains bad syntax or cannot}\\
&{be fulfilled}\\
\hline
\textbf{5xx: Server Error}&{The server failed to fulfill an apparently}\\
&{valid request}\\
\hline
\end{tabular}
\end{table}
\begin{table}[h]
\centering
\footnotesize
\begin{tabular}{|r|l|}
\multicolumn{2}{c}{\textbf{Known service code}}\\
\hline
\textbf{200}&{OK}\\
\hline
\textbf{201}&{Created}\\
\hline
\textbf{202}&{Accepted}\\
\hline
\textbf{204}&{No Content}\\
\hline
\textbf{301}&{Moved Permanently}\\
\hline
\textbf{302}&{Moved Temporarily}\\
\hline
\textbf{304}&{Not Modified}\\
\hline
\textbf{400}&{Bad Request}\\
\hline
\textbf{401}&{Unauthorized}\\
\hline
\textbf{403}&{Forbidden}\\
\hline
\textbf{404}&{Not Found}\\
\hline
\textbf{500}&{Internal Server Error}\\
\hline
\textbf{501}&{Not Implemented}\\
\hline
\textbf{502}&{Bad Gateway}\\
\hline
\textbf{503}&{Service Unavailable}\\
\hline
\end{tabular}
\end{table}

\vspace{10cm}
\section{Examples}
The following pieces of code are examples of TCP client connection to \textbf{www.google.it}, using functions explained in Chapter \ref{networkC}.
\subsection{HTTP 0.9}
The following piece of code define a structure, used to connect to Google server. 

The most important thing is that \textbf{socket()} is entry-point for level 4, but also \textbf{connect()} is the request to Kernel to extablish the connection.\\ \textbf{read()} and \textbf{write()} are system calls used respectively to obtain result(response) of a request and to generate request.\\ These function permit us to ask to lower level to do this things, without knowing content of system buffers (stream). The second part is only used to read the input.

\subsection{HTTP 1.0}
The protocol has no mandatory headers to be added in the request field. This protocol is compliant with HTTP 0.9.
To keep the connection alive, "Connection" header with "keep-alive" as header field must be added to request message. The server, receiving the request, replies with a message with the same header value for "Connection".\\
This is used to prevent the closure of the connection, so if the client needs to send another request, he can use the same connection.
This is usually used to send many files and not only one.\\
The connection is kept alive until either the client or the server decides that the connection is over and one of them drops the connection. If the client doesn't send new requests to the server, the second one usually drops the connection after a couple of minutes.\\
The client could read the response of request, with activated keep alive option, reading only header and looking to "Content-length" header field value to understand the length of the message body. This header is added only if a request with keep-alive option is done.\\
This must be done because we can't look only to empty system stream, because it could be that was send only the response of the first request or a part of the response.\\
Otherwise, when the option keep alive is not used, the client must fix a max number of characters to read from the specific response to his request, because it doesn't know how many character compose the message body.

\subsection{HTTP 1.1}
It has by default the option keep alive actived by default with respect to HTTP 1.0. It has the mandatory header "Host" followed by the hostname of the remote system to which the request or the response is sent. The body is organized in chunks, so we need the connection kept alive to manage future new chunks.\\
This is useful with dynamic pages, in which the server doesn't know the length of the stream in advance and can update the content of the stream during the extablished connection, sending a fixed amount of bytes to client. We can check if the connection is chunked oriented, looking for the header "Transfer-Encoding" with value "chunked".\\ 
Each connection is composed by many chunks and each of them is composed by chunk length followed by chunk body, except for the last one that has length 0 (see Figure \ref{chunked_body}). The following grammar represents how the body is organized:
\begin{center}
\begin{tabular}{c}
\begin{lstlisting}[linewidth=320pt, basicstyle=\footnotesize\sffamily,]
Chunked-Body   = *chunk
                 last-chunk
                 trailer
                 CRLF

chunk          = chunk-size [ chunk-extension ] CRLF
                 chunk-data CRLF

chunk-size     = 1*HEX
last-chunk     = 1*("0") [ chunk-extension ] CRLF

chunk-extension= *( ";" chunk-ext-name [ "=" chunk-ext-val ] )

chunk-ext-name = token
chunk-ext-val  = token | quoted-string
chunk-data     = chunk-size(OCTET)
trailer        = *(entity-header CRLF)
\end{lstlisting}
\end{tabular}
\end{center}

\begin{figure}[h]
\centering
\includegraphics[scale=0.5]{Images/HTTP/Chunked-Body}\caption{\footnotesize{Chunked body.}}\label{chunked_body}
\end{figure}
\chapter{Resolution of names}
The following section will talk about history of technologies under the resolution of server names in URL to their IP addresses, needed to extablish the connection.

\section{Network Information Center (NIC)}\label{NIC_section}
This type of architecture was used in the past to resolve names. Each client has its own file \textbf{HOSTS.txt}, with resolution of names. The client shared its file with a central system, called \textbf{NIC} (Figure \ref{NIC}).\\
This system collects all the files, like an hub, and shared resolution names to other clients (Figure \ref{NIC}).\\
This architecture is unfeasable and not scalable with nowadays number of IP addresses, because the files become very huge and transfering becomes very slow.\\
\begin{figure}[h]
\centering
\includegraphics[scale=0.4]{Images/Resolution/NIC}
\caption{\footnotesize{How NIC worked.}}\label{NIC}
\end{figure}


\section{Domain Name System (DNS)}\label{DNS_system}
The file \textbf{HOSTS.txt}\ref{NIC_section} is yet used in nowadays UNIX systems. The specified host name is searched in local \textbf{/etc/hosts.txt}, that contains local and private addresses resolution table, and if not found, it will be searched through DNS\cite{RFC1034}.

\subsection{Goals}
\begin{enumerate}
\item{Names should not be required to contain network identifiers, addresses, routes, or similar information as part of the name.}
\item{The sheer size of the database and frequency of updates suggest that it must be maintained in a distributed manner, with local caching to improve performance.\\
Approaches that attempt to collect a consistent copy of the entire database will become more and more expensive and difficult, and hence should be avoided.\\
The same principle holds for the structure of the name space, and in particular mechanisms for creating and deleting names; these should also be distributed.}
\item{Where there are tradeoffs between the cost of acquiring data, the speed of updates, and the accuracy of caches, the source of the data should control the tradeoff.}
\item{The costs of implementing such a facility dictate that it be generally useful, and not restricted to a single application.\\
We should be able to use names to retrieve host addresses, mailbox data, and other as yet undetermined information. All data associated with a name is tagged with a type, and queries can be limited to a single type.}
\item{Because we want the name space to be useful in dissimilar networks and applications, we provide the ability to use the same name space with different protocol families or management.\\
For example, host address formats differ between protocols, though all protocols have the notion of address. The DNS tags all data with a class as well as the type, so that we can allow parallel use of different formats for data of type address.}
\item{We want name server transactions to be independent of the communications system that carries them.\\
Some systems may wish to use datagrams for queries and responses and only establish virtual circuits for transactions that need the reliability (e.g., database updates, long transactions); other systems will use virtual circuits exclusively.}
\item{The system should be useful across a wide spectrum of host capabilities.\\
Both personal computers and large timeshared hosts should be able to use the system, though perhaps in different ways.}
\end{enumerate}

\subsection{Hierarchy structure}
Hierarchy permits to manage a lot of nambers of domain names and IP addresses, reducing the time spent to resolve them. Given for example the host name \textbf{www.dei.unipd.it}, we have a \textbf{Name Server (NS)} for each of the domain name inside it (Figure \ref{DNS_hierarchy}).\\
The tree hierarchy has a name server for each one of its internal nodes.The name server gives us only the name of the name server of the lower level to which we need to go.\\
To obtain the IP address of this name server, we need to ask, to name server of upper layer, a \textbf{glue record}. The glue record is an additional information that needs us to understand how to rich that name server. Hence the glue record is the IP address of NS of the lower level in hierarchy. \\
For each request to NS, we obtain also the expiration time information because a caching approach is adopted also in DNS but at level 4. There are 13 root name servers that are returned when asking resolution to root. \\
In reality root name servers are more than 13 but the communication used in DNS is made through UDP and this type of connection supports only 13 simultaneously transfering. The local DNS server for the device, managed by my network provides, contains the 13 root servers and permit us to reach at least one DNS root server.\\
The 13 DNS root servers are added locally at its installation of local DNS and updated assuming that at least one root server of them can be reachable. There is no address record for the root.\\
In general structure of the queries to name servers, we ask only the domain resolution for a domain that composes name \ref{default_DNS}.\\
To use a caching system efficiently, we need to make a recursive query, sending the request of resolution of the whole name with all its domains (Figure \ref{extreme_recursive}. All the name servers, where the query passes throuh, store information about resolution. This system is never applied as it is.\\
In reality we use an hybrid version, that uses partially recursion \ref{partial_recursion}. Local DNS  usually has huge cache with main important names and also first and second level have caches. So local DNS rarely asks resolution to TOP Level Domain or Root.\\
Recursive query option in dig command is made by a flag default set yes and that is used in UDP packet as an additional information. The Root Name Server decides if it wants to accept recursive query or if not, how mmany domains can resolve.\\
I can group some domains, defining a zone, so I can use only a name server for the zone that solves more domains together \ref{DNS_zone}. So the name servers are authorithative over zones and not only single domains. It depends if domains are grouped or not.\\
The creation of the zones are used to manage easily the responsability and the organization over the zones, partitioning them. Another reason for this partition in zones is because of some domains has few names and so it's better to group the domain with others.
\begin{figure}[h]
\centering
\includegraphics[scale=0.4]{Images/Resolution/DNS_hierarchy}
\caption{\footnotesize{DNS structure.}}\label{DNS_hierarchy}
\end{figure}
\begin{lstlisting}[linewidth=470pt, style=code, caption=Example of default DNS queries using dig.]
//Ask for root name server to the default name server 
dig -t NS -n .
 
//Ask for address of root name server "a.root-servers.net", previously chosen
dig -t A -n a.root-servers.net 

//Ask for "it" name server to the "a.root-servers.net" address, previously chosen
dig @198.41.0.4 -t NS it 

//Ask for address of "nameserver.cnr.it" name server,  previously chosen for "it" domain
dig @198.41.0.4 -t A  nameserver.cnr.it

//Ask for "unipd.it" name server to the "nameserver.cnr.it" address 
dig @194.119.192.34 -t NS -n unipd.it

//Ask for "unipd.it" name server to the "nameserver.cnr.it" address 
dig @194.119.192.34 -t A unipd.it

//Ask for "dei.unipd.it" name server to one ("mail.dei.unipd.it" 
dig @147.162.1.100 -t NS  dei.unipd.it

//Ask for address of "mail.dei.unipd.it" name server, previously chosen 
dig @147.162.1.2 -t A  mail.dei.unipd.it 

//Ask for address of "www.dei.unipd.it" to "mail.dei.unpd.it" name server, previousy chosen 
dig @147.162.2.100 -t A  www.dei.unipd.it
\end{lstlisting} 
\begin{figure}[h]
\centering
\includegraphics[scale=0.4]{Images/Resolution/default_DNS}
\caption{\footnotesize{Default DNS behaviour without caching.}}\label{default_DNS}
\end{figure}
\begin{figure}[h]
\centering
\includegraphics[scale=0.4]{Images/Resolution/default_DNS}
\caption{\footnotesize{Completely recursive DNS structure.}}\label{recursive_DNS}
\end{figure}
\begin{figure}[h]
\centering
\includegraphics[scale=0.4]{Images/Resolution/hybrid_DNS}
\caption{\footnotesize{Hybrid DNS structure.}}\label{hybrid_DNS}
\end{figure}
\begin{figure}[h]
\centering
\includegraphics[scale=0.4]{Images/Resolution/DNS_zone}
\caption{\footnotesize{Example of partitioning into zones.}}\label{DNS_zone}
\end{figure}
\begin{comment}
\chapter{HTML}
The body of an HTTP request, it's often composed by the HTML related page. Each click, of a link inside the web page, generates a new request to the server with GET method.
\input{Chapters/CSS}
\input{Chapters/Javascript}
\input{Chapters/PHP}
\end{comment}
\chapter{Shell}

\section{Commands}

\begin{table}[h]
\begin{tabular}{|l|l|l|}
\hline
\multicolumn{2}{|l|}{\textbf{strace} objFile} & {List all the system calls used in the program.}\\
\hline
\multicolumn{2}{|l|}{\textbf{gcc} -o objFile source \textbf{-v}} & {List all the path of libraries and headers used in creation of objFile.}\\
\hline
\multirow{3}{*}{\textbf{netstat}} & {-t} & {List all the active TCP connections showing domain names.}\\
\cline{2-3}
& {-u} & {List all the active UDP connections showing domain names.}\\
\cline{2-3}
& {-n} & {List all the active, showing IP and port numbers.}\\
\hline
\multicolumn{2}{|l|}{\textbf{nslookup} domain} & {Shows the IP address related to the domain (E.g. IP of www.google.it)}\\
\hline
\end{tabular}
\end{table}

\section{Files}

\begin{table}[h]
\begin{tabular}{|l|l|}
\hline
\multirow{2}{*}{\textbf{/etc/services}} & {List all the applications with their port}\\
& {and type of protocol (TCP/UDP).}\\
\hline
{\textbf{/usr/include/x86\_64-linux-gnu/bits/socket.h}} & {List all the protocol type possible for socket.}\\
\hline
{\textbf{/usr/include/x86\_64-linux-gnu/sys/socket.h}} & {Definition of struct sockaddr and specific ones.}\\
\hline
{\textbf{}} & {.}\\
\hline
\end{tabular}
\end{table}

\section{vim}
\subsection{.vimrc}
In this section there will be shown the file \textbf{.vimrc} that can be put in the user home (\textbf{$\sim$} or \textbf{\$HOME} or \textbf{--}) or in the path \textbf{/usr/share/vim/} to change main settings of the program.

\lstinputlisting[caption={\footnotesize{web\_client.c}}, style=code, firstnumber=1, firstline=1, lastline=8, label=web_client, language=c]{Code/vimrc}

\subsection{Shortcuts}

%\section{Usefull information about shell}
\input{Chapters/Vim}
%Bibliography part (called References)
\addcontentsline{toc}{chapter}{References}
\bibliographystyle{plain}
\renewcommand{\bibname}{References}
\bibliography{Chapters/biblio}
\end{document}